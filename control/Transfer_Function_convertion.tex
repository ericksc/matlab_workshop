% This LaTeX was auto-generated from MATLAB code.
% To make changes, update the MATLAB code and export to LaTeX again.

\documentclass{article}

\usepackage[utf8]{inputenc}
\usepackage[T1]{fontenc}
\usepackage{lmodern}
\usepackage{graphicx}
\usepackage{color}
\usepackage{listings}
\usepackage{hyperref}
\usepackage{amsmath}
\usepackage{amsfonts}
\usepackage{epstopdf}
\usepackage{matlab}

\sloppy
\epstopdfsetup{outdir=./}
\graphicspath{ {./Transfer_Function_convertion_images/} }

\begin{document}

\section*{Conversión de representación de sistemas}

\subsection*{Función de transferencia basado en seleción de polos y los residuos}

\begin{par}
\begin{flushleft}
Conociendo la información de los polos y ceros del sistema se puede obtener su correspondiente función de transferencia.
\end{flushleft}
\end{par}

\begin{matlabcode}
r = [-6 -4 3];
p = [-3 -2 -1];
k = 1;
[num,den] = residue(r,p,k)
\end{matlabcode}
\begin{matlaboutput}
num = 
     1    -1    -8     0

den = 
     1     6    11     6

\end{matlaboutput}
\begin{matlabcode}
printsys(num,den,'s')
\end{matlabcode}
\begin{matlaboutput}
 
num/den = 
 
      s^3 - 1 s^2 - 8 s
   ----------------------
   s^3 + 6 s^2 + 11 s + 6
\end{matlaboutput}

\subsection*{Obtención de polos y ceros apartir de función de transferencia}

\begin{matlabcode}
num = [1,-1,-8,0]
\end{matlabcode}
\begin{matlaboutput}
num = 
     1    -1    -8     0

\end{matlaboutput}
\begin{matlabcode}
den = [1,6,11,6]
\end{matlabcode}
\begin{matlaboutput}
den = 
     1     6    11     6

\end{matlaboutput}
\begin{matlabcode}

[z,p,K] = tf2zp(num,den)
\end{matlabcode}
\begin{matlaboutput}
z = 
         0
    3.3723
   -2.3723

p = 
   -3.0000
   -2.0000
   -1.0000

K = 1
\end{matlaboutput}

\end{document}
